\documentclass[a4paper,12pt]{article}
\usepackage[utf8]{inputenc}
\usepackage{graphicx}
\usepackage{hyperref}
\usepackage{amsmath, amssymb}
\usepackage{authblk}


\title{Requirement Engineering and Design Project \\ SE4HPC_RDproject} 
\author[1]{\\Giovanni La Gioia}
\author[2]{\\Luca Leonzio}
\author[3]{\\Matteo Parimbelli}
\author[4]{\\Serena Tolla}
\affil[1,2,3]{Politecnico di Milano}
\date{\today}

\begin{document}

\maketitle
\tableofcontents
\newpage

\documentclass[a4paper,12pt]{article}
\usepackage[utf8]{inputenc}
\usepackage{graphicx}
\usepackage{hyperref}
\usepackage{amsmath, amssymb}

\title{Requirement Engineering and Design Project \\ SE4HPC_RDproject}
\author{Serena Tolla \\ Politecnico di Milano}
\date{\today}

\begin{document}

\maketitle
\tableofcontents
\newpage

% SECTION 1 - Introduction
\section{Introduction}
This document describes the **SustainCity** project, developed for the **Software Engineering for HPC** course.
The objective is to analyze urban mobility management to optimize sustainability, reducing traffic and environmental impact.

\subsection{Goals}
The main goals of the project are:
\begin{itemize}
    \item Monitoring traffic flows and dynamically adapting traffic light durations.
    \item Analyzing traffic patterns to optimize urban mobility and public transportation.
    \item Managing events that attract large crowds to minimize road congestion.
\end{itemize}

\newpage

% SECTION 2 - Requirements Analysis
\section{Requirements Analysis}
\subsection{Actors involved}
The key actors in the system include:
\begin{itemize}
    \item \textbf{Traffic sensors}, which measure crossing times at intersections.
    \item \textbf{Public transportation microservices}, which provide schedules and availability.
    \item \textbf{Information channels}, which broadcast news on urban events.
    \item \textbf{Citizens and urban area managers}, involved in the analysis and application of solutions.
\end{itemize}

\subsection{Functional requirements}
\begin{itemize}
    \item The system must receive up-to-date data on traffic flows.
    \item It must analyze patterns and suggest optimizations for road configurations.
    \item It must notify urban managers about possible interventions.
\end{itemize}

\newpage

% SECTION 3 - System Architecture
\section{System Architecture}
\subsection{Component Diagrams}
The system consists of various units, including:
\begin{itemize}
    \item \textbf{Traffic analysis module}, which collects data and identifies patterns.
    \item \textbf{Traffic light configuration module}, which adapts signal timing.
    \item \textbf{Event management module}, which suggests road modifications based on events.
\end{itemize}

\subsection{Sequence Diagrams}
For each key action in the system, sequence diagrams are defined to describe the interaction between components.

\newpage

% SECTION 4 - Implementation and Expected Results
\section{Implementation and Expected Results}
The implementation will be managed through a GitHub repository organized with the following folders:
\begin{itemize}
    \item \textbf{docs/} - Contains the LaTeX document and project specifications.
    \item \textbf{src/} - Includes the source code written in C++.
    \item \textbf{diagrams/} - Contains UML diagrams and other technical representations.
\end{itemize}

\subsection{Evaluation Metrics}
The success of the system will be evaluated based on the following parameters:
\begin{itemize}
    \item Reduction of traffic congestion.
    \item Optimization of public transportation resources.
    \item Acceptance of configuration proposals by urban authorities.
\end{itemize}

\end{document}
