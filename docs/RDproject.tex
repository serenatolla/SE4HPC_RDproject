\documentclass[a4paper,12pt]{article}
\usepackage[utf8]{inputenc}
\usepackage{graphicx}
\usepackage{hyperref}
\usepackage{amsmath, amssymb}
\usepackage{authblk}


\title{Requirement Engineering and Design Project \\ SE4HPC_RDproject}
\vspace{3cm} % Adds space between title and authors

\author[1]{Giovanni La Gioia}
\vspace{0.5cm} % Space between authors
\author[2]{Luca Leonzio}
\vspace{0.5cm}
\author[3]{Matteo Parimbelli}
\vspace{0.5cm}
\author[4]{Serena Tolla}
\affil[1,2,3,4]{Politecnico di Milano}
\date{Last update: \today}

\begin{document}
\maketitle
\newpage
\tableofcontents
\clearpage
\newpage

% SECTION 1 - The Project and Its Goals
\section{The Project and Its Goals}
This document describes the \textbf{SustainCity} project, developed as part of the \textbf{Software Engineering for HPC} course. Its objective is to tackle urban mobility challenges using \textbf{real-time traffic management, event-driven optimizations, and public transport coordination}. By leveraging \textbf{dynamic infrastructure adaptations}, the system aims to \textbf{reduce congestion, improve sustainability}, and \textbf{enhance traffic flow efficiency} in urban areas.

We structured this section into three key parts:
\begin{itemize}
    \item The \textbf{Preface}, which introduces the broader environmental and urban challenges motivating this project.
    \item The \textbf{Problem Posed}, detailing the technical and societal issues to be addressed.
    \item The \textbf{Project Goals}, outlining the core functionalities and expected outcomes.
\end{itemize}

\subsection{Preface}
Urban mobility inefficiencies significantly contribute to \textbf{air pollution, congestion, and greenhouse gas emissions}. Despite the presence of \textbf{public transportation}, many cities still face considerable challenges in \textbf{traffic optimization} and \textbf{event-based urban planning}.

To address these concerns, the \textbf{SustainCity} project integrates \textbf{traffic monitoring, event planning, and transport analytics} into an adaptive system capable of making \textbf{data-driven recommendations} for \textbf{real-time city infrastructure improvements}.

\subsection{Problem Posed}
The \textbf{SustainCity} system aims to optimize urban traffic using three key strategies:

\begin{itemize}
    \item \textbf{Type1 Actions (Real-time Traffic Optimization)} 
    The system dynamically adjusts traffic light durations based on live congestion data.  

    \item \textbf{Type2 Actions (Long-term Traffic Pattern Analysis)}  
    The system analyzes daily congestion patterns** to recommend permanent adjustments, such as one-way street configurations, adaptive traffic signal schedules, and public transport optimizations.

    \item \textbf{Type3 Actions (Event-driven Traffic Management)}
    SustainCity monitors city event feeds (e.g., concerts, fairs, sports games) and adjusts road and transport infrastructure dynamically to reduce congestion during high-footfall gatherings.
\end{itemize}

\noindent Additionally, SustainCity generates \textbf{reports for urban planners and citizens}:
\begin{itemize}
    \item \textbf{Daily Reports}: Summarizing average traffic flows and actions taken (Type1).
    \item \textbf{Yearly Reports}: Documenting major infrastructure proposals (Type2, Type3) and their implementation status (accepted/rejected).
\end{itemize}

\subsection{Project Goals}
To improve urban traffic management, SustainCity focuses on:
\begin{itemize}
    \item \textbf{Optimizing real-time traffic signals} using congestion data (Type1).
    \item \textbf{Enhancing urban infrastructure and transport schedules} (Type2).
    \item \textbf{Minimizing congestion during major city events} (Type3).
    \item \textbf{Generating actionable insights for city planners and the public}.
\end{itemize}

The project is designed to be \textbf{scalable, adaptive, and fully integrated} with existing urban mobility systems. By providing \textbf{data-driven recommendations}, SustainCity aims to create \textbf{efficient, sustainable urban transport solutions}.

\newpage

% SECTION 2 - Requirement Analysis
\section{Requirement Analysis}
\subsection{Relevant Human and Non-Human Actors}
Key actors involved in the system include:
\begin{itemize}
    \item \textbf{Traffic sensors} – Measure crossing times at intersections.
    \item \textbf{Public transport microservices} – Provide transit schedules.
    \item \textbf{Urban managers} – Implement solutions based on system suggestions.
    \item \textbf{Information channels} – Broadcast event updates impacting mobility.
\end{itemize}

\subsection{Use Cases}
The system manages urban mobility challenges using:
\begin{itemize}
    \item Dynamic traffic light adjustments.
    \item Real-time event awareness for road optimization.
    \item Predictive congestion analysis.
\end{itemize}

\subsection{Domain Assumptions}
\begin{itemize}
    \item The city collects and shares traffic data reliably.
    \item Event organizers notify the system about major public events.
    \item Authorities consider proposed adjustments for traffic flow improvement.
\end{itemize}

\subsection{Requirements}
\subsubsection{Functional Requirements}
\begin{itemize}
    \item The system must collect and process real-time traffic data.
    \item It should provide optimization recommendations based on traffic models.
    \item Integration with public transport schedules must be supported.
\end{itemize}

\subsubsection{Non-Functional Requirements}
\begin{itemize}
    \item High availability for real-time data processing.
    \item Low-latency response times to traffic pattern changes.
    \item Secure access for city planners and mobility experts.
\end{itemize}

\newpage

% SECTION 3 - Design
\section{Design}
\subsection{General Description of the Architecture}
The system consists of multiple components:
\begin{itemize}
    \item \textbf{Traffic Analysis Module} – Collects and processes mobility data.
    \item \textbf{Optimization Engine} – Suggests adjustments in urban traffic.
    \item \textbf{Event Management Interface} – Monitors large events impacting the city.
\end{itemize}

\subsection{Sequence Diagrams}
Below are three key sequence diagrams illustrating critical interactions:
\begin{itemize}
    \item Traffic sensor data integration and optimization.
    \item Dynamic adjustments in traffic signals based on congestion prediction.
    \item Real-time notifications for urban managers during major city events.
\end{itemize}

\subsection{Critical Points and Design Decisions}
\begin{itemize}
    \item The real-time nature of traffic adjustments requires low-latency computation.
    \item System must adapt dynamically to unexpected urban events.
    \item Security measures for ensuring data integrity and privacy.
\end{itemize}

\end{document}
