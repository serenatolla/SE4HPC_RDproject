\documentclass[a4paper,12pt]{article}
\usepackage[utf8]{inputenc}
\usepackage{graphicx}
\usepackage{hyperref}
\usepackage{amsmath, amssymb}
\usepackage{authblk}

\title{SustainCity: \\ The Future of Adaptive Urban Movement \\ v 1.0.1} % Thought of this but if u have better ideas 
\vspace{3cm} % Adds space between title and authors

\author[1]{Giovanni La Gioia}
\vspace{0.5cm} % Space between authors
\author[2]{Luca Leonzio}
\vspace{0.5cm}
\author[3]{Matteo Parimbelli}
\vspace{0.5cm}
\author[4]{Serena Tolla}
\affil[1,2,3,4]{Politecnico di Milano}
\date{Last update: \today}

\begin{document}
\maketitle
\newpage
\tableofcontents
\clearpage
\newpage

% SECTION 1 - The Project and Its Goals
\section{The Project and Its Goals}
The following describes the \textbf{SustainCity} project and its goals to be achieved, developed as part of the \textbf{Software Engineering for HPC} course. We structured this first section into three key parts:
\begin{itemize}
    \item The \textbf{Preface}, introducing the broader environmental and urban challenges motivating this project.
    \item The \textbf{Problem Posed}, describing in detail the technical and society issues to be addressed.
    \item The \textbf{Project Goals}, outlining the core functionalities and expected outcomes.
\end{itemize}

\subsection*{Preface}
Two urgent global concerns are environmental sustainability and climate change; because of air pollution and greenhouse gas emissions, transportation - especially urban commuting - worsens those issues. 
\\ \\ The objective of the SustainCity project is to face urban mobility challenges through real-time traffic management, event-driven optimizations, and public transport coordination. 
\\ \\To address these concerns, the project integrates traffic monitoring, event planning, and transport analytics into an adaptive system capable of creating data-driven recommendations for real-time city infrastructure improvements. 
\\ \\Thanks to these system recommendations, urban commuting can be drastically reduced, improving both citizens’ well-being and environmental sustainability.

\subsection*{Problem Statement}
The \textbf{SustainCity} system aims to optimize urban traffic using three key strategies:
\begin{itemize}
    \item \textbf{Real-time Traffic Optimization (Type1 Actions)} 
    The system continuously adapts traffic light timings using real-time congestion data to optimize flow.  
    \item \textbf{Long-term Traffic Pattern Analysis (Type2 Actions)}  
    The system analyzes daily congestion patterns to recommend permanent adjustments, such as one-way street configurations, adaptive traffic signal schedules, and public transport optimizations.
    \item \textbf{Event-driven Traffic Management (Type3 Actions)}
    SustainCity monitors city event feeds (e.g., concerts, fairs, sports games) and adjusts road and transport infrastructure dynamically to reduce congestion during high-footfall gatherings.
\end{itemize}

\noindent Additionally, SustainCity generates \textbf{reports for urban planners and citizens}:
\begin{itemize}
    \item \textbf{Daily Reports}: Summarizing average traffic flows and actions taken (Type1).
    \item \textbf{Yearly Reports}: Documenting major infrastructure proposals (Type2, Type3) and their implementation status (accepted/rejected).
\end{itemize}

\noindent To accomplish these goals, we can rely on the following elements: 
\begin{itemize}
    \item A preexisting infrastructure offering sensors that measure the number of seconds cars need to cross each intersection. This infrastructure is developed according to the event-based style. All sensors periodically publish the data they have acquired on a message bus. 
    \item A microservice offering information about public transport schedules. In particular, this microservice offers the following operations: 
    \begin{itemize} 
        \item \textbf{getScheduleByStreet}: Given the name of a street, it returns the timetable of all stops present on that street. 
        \item \textbf{getScheduleByLine}: Given the number of a specific line, it returns the complete timetable for that line.   
    \end{itemize}
    \item A news channel that transmits information about the events in the city as soon as they are planned.
\end{itemize}

\subsection*{Project Goals}
Our main goal was to develop a software system that creates efficient, sustainable urban transport solutions by providing data-driven recommendations. Therefore, the document aims to meet two objectives: 
\begin{itemize}
    \item Ensuring all functional and technical requirements are met.
    \item Designing a scalable, efficient, and well-structured urban-mobility solution.
\end{itemize}

\newpage

% SECTION 2 - Requirement Analysis
\section{Requirement Analysis}
\subsection{Relevant Human and Non-Human Actors}
Let's begin by examining the \textit{human actors} involved in this system first:
\begin{itemize}
    \item \textbf{Traffic Engineers} (Type1): They supervise and verify the correct functioning of the automatic traffic light regulation system. They take action in the event of malfunction, validate traffic management strategies, and analyze logs for future improvements.
    \item \textbf{Urban Area Manager} (Type1, Type2, Type3): They are responsible for overseeing urban infrastructure, long-term policies and constraints related to dynamic traffic light regulation. They also evaluate traffic optimization proposals and approve or reject recommendations related to traffic configurations, transportation scheduling, and road management. Their decisions directly impact the effectiveness of SustainCity's long-term strategies.
    \item \textbf{City Planners and Officials} (Type 1, Type2, Type3): Work alongside \textbf{Urban Area Managers} to assess traffic data reports, evaluate permanent infrastructure adaptations, and implement policy changes based on recommendations. They ensure that adjustments are in accordance with environmental and urban regulations. They may also be involved in the approval and monitoring of automated traffic regulation systems.
    \item \textbf{Citizens}(Type1, Type2, Type3): Although not actively involved in the system’s technical workings, they are end-users who benefit from improved traffic conditions, congestion reduction, and optimized public transport schedules. They receive notifications through city mobility platforms, public dashboards, transport apps, and digital road signs.
\end{itemize}
\noindent On the other hand, there are also several \textit{non-human actors} needed to ensure the system operates efficiently and delivers accurate data-driven insights:
\begin{itemize}
    \item \textbf{Traffic Management System} (Type1, Type2, Type3): Acts as the core analytical component by collecting, processing, and interpreting data from various sources, such as traffic sensors. This actor continuously monitors daily traffic flows, identifies congestion hot spots, and detects recurring traffic patterns that could benefit from optimization.
    \item \textbf{Public Transport Microservice} (Type2, Type3): Provides comprehensive and real-time public transport schedules. By integrating this microservice, the system can propose modifications to transport schedules, aligning them with identified traffic patterns to optimize commuter convenience and reduce congestion.
    \item \textbf{Sensors} (Type1, Type2, Type3): Physical devices deployed at intersections and major roadways; these actors gather real-time data about traffic conditions. Accurate and timely sensor data is foundational, enabling the system to analyze traffic patterns and propose meaningful optimizations reliably.
    \item \textbf{Event Monitoring System} (Type3): Detects and transmits event schedules (concerts, sports games, fairs), triggering event-driven traffic adjustments and real-time mobility recommendations. It works closely with the \textbf{Traffic Management System} and \textbf{Public Transport Microservice} to prevent congestion.
    \item \textbf{Notification System} (Type1, Type2, Type3): A critical component for disseminating structured reports and real-time alerts to all relevant actors. It ensures that traffic management decisions — whether real-time, event-driven, or long-term — are properly communicated to all relevant stakeholders.
\end{itemize}

\newpage

\section{Use Cases}
The following use cases describe how the system interacts with traffic data, urban managers, and external events to achieve its objectives. Each use case type corresponds to a specific level of system autonomy and decision-making: Type1 actions are fully automated, Type2 involve semi-automated analysis with human validation, and Type3 relate to exceptional or event-driven scenarios requiring coordinated interventions. The use cases are organized into tables for clarity.

\subsection*{Type1 Use Cases}
\begin{table}[h!]
\centering
\begin{tabular}{|p{4.5cm}|p{8.5cm}|}
\hline
\textbf{Use Case} & \textbf{Description} \\
\hline
Real-Time Traffic Monitoring & The system continuously receives data from traffic sensors at key intersections, monitoring vehicle wait times and traffic density. \\
\hline
Dynamic Light Adjustment Decision & Based on thresholds or patterns, the system autonomously decides to alter green light durations using predefined rules or adaptive algorithms. \\
\hline
Actuation of Traffic Light Changes & Once a decision is made, the system dispatches updated timing configurations to traffic lights for short-term application. \\
\hline
\end{tabular}
\caption{Type1 Use Cases}
\end{table}

\subsubsection*{Use Case 1: Real-Time Traffic Monitoring}
\textbf{Primary Actors}: Traffic Sensors \\
\textbf{Supporting Actors}: Traffic Management System \\
\textbf{Use case flow}: 
\begin{itemize}
    \item \textit{Collect Data} Traffic sensors continuously measure real-time vehicle wait times and traffic density at intersections and publish this data on the event bus.
    \item \textit{Receive and Process Data} The Traffic Management System continuously receives and validates incoming sensor data, checking for data consistency and accuracy.
    \item \textit{Store and Document Data} The processed data is stored internally for immediate access by other modules within the Traffic Management System.
\end{itemize}
\textbf{Assumptions}: Traffic Sensors are fully operational, event bus system is reliable and timely \\
\textbf{Requirements}: continuously collect real-time sensor  data at intersections (FR1), data collected must be accurate and timely (FR2) \\

\subsubsection*{Use Case 2: Dynamic Light Adjustment Decision}
\textbf{Primary Actors}: Traffic Management System \\
\textbf{Supporting Actors}: Traffic Sensors, Notification System \\
\textbf{Use case flow}: 
\begin{itemize}
    \item \textit{Evaluate Traffic Conditions} Traffic Management System evaluates received sensor data to identify congestion or imbalance between intersecting roads.
    \item \textit{Decision Making} Based on predefined thresholds and adaptive algorithms, the system autonomously decides if traffic light timing adjustments are necessary.
    \item \textit{Prepare Adjustment Instructions} System formulates the specific adjustment parameters (timing alterations) required for each affected intersection.
\end{itemize}
\textbf{Assumptions}: pre-defined thresholds and adaptive algorithms are accurate and reliable, real-time processing capabilities exist within the system \\
\textbf{Requirements}: autonomously detect traffic imbalances (FR1), clearly document each decision and its justification (FR2) \\

\subsubsection*{Use Case 3: Actuation of Traffic Light Changes}
\textbf{Primary Actors}: Traffic Management System \\
\textbf{Supporting Actors}: Traffic Controllers, Notification System \\
\textbf{Use case flow}: 
\begin{itemize}
    \item \textit{Dispatch Adjustment Commands} System sends calculated timing configurations directly to the traffic controllers managing the physical lights.
    \item \textit{Apply Configuration Immediately} Traffic controllers immediately apply the new timing settings without human intervention.
    \item \textit{Log Action} Traffic Management System logs the applied changes, including time of application, intersections affected, and rationale for transparency and audit purposes.
    \item \textit{Public Notification} Notification System optionally communicates any critical short-term adjustments to commuters via dashboards if necessary.
\end{itemize}
\textbf{Assumptions}: reliable, uninterrupted communication between management system and traffic controllers \\
\textbf{Requirements}: autonomous dispatch and immediate application of new traffic timings (FR1), comprehensive logging of every applied change for auditing (FR2) \\

\newpage

\subsection*{Type2 Use Cases}

\begin{table}[h!]
\centering
\begin{tabular}{|p{4.5cm}|p{8.5cm}|}
\hline
\textbf{Use Case} & \textbf{Description} \\
\hline
Traffic Pattern and Public Transport Analysis & The system processes daily traffic and public transport data to detect recurring congestion patterns, bottlenecks, or inefficiencies. \\
\hline
Suggest Optimizations & Based on analysis, the system recommends traffic light adjustments, road direction changes, and transit schedule modifications. \\
\hline
Review and Apply Changes & Urban area managers review and approve or reject system-generated suggestions, implementing selected actions. All modifications are published on User App and maps are modificated. \\
\hline
\end{tabular}
\caption{Type2 Use Cases}
\end{table}

\subsubsection*{Use Case 1: Traffic Pattern and Public Transport Analysis}
\textbf{Primary Actors}: Traffic Sensors, Public Transport Microservice \\
\textbf{Supporting Actors}: Traffic Management System \\ 
\textbf{Use case flow}: 
\begin{itemize}
    \item \textit{Collect Data} Traffic sensors continuously measure the number of seconds vehicles take to cross intersections and periodically publish this data on the event bus.
    \item \textit{Process Traffic Data} The system processes incoming sensor data, filtering out anomalies and organizing it into usable information.
    \item \textit{Store Analysis Results} The data collected are stored internally for use by the Optimization Engine. The data provided must be usable for the Traffic Management System.
\end{itemize}
\textbf{Assumptions}: Traffic sensors and Public Transport Microservice always working, enough storage.
\textbf{Requirements}: the system shall collect real-time data from city-wide traffic sensor and public transport microservice (FR1), the Sensors must produce reliable data and usable data for Traffic Management System (FR2), the system shall store and document all data elaborating results internally for subsequent use (FR3).

\subsubsection*{Use Case 2: Suggest Optimizations}
\textbf{Primary Actors}: Traffic Management System \\
\textbf{Supporting Actors}:  Urban Area, Traffic Sensors, Public Transport Microservice \\
\textbf{Use case flow}: 
\begin{itemize}
\item \textit{Retrieve Analysis Results}
The Optimization Engine retrieves combined traffic pattern and public transport data results previously generated by the sensors and the microservice.
\item \textit{Data refinement} The Traffic Management System apply techniques in order to have data ready to be processed for Machine Learning techniques both for Supervised and Unsupervised data.
\item \textit{Generate Optimization Recommendations} The system generates a detailed list of specific, actionable optimization suggestions, including: recommended directional changes for particular road segments, proposed timetable modifications for bus and tram services.
\item \textit{Prioritize and Store Recommendations} The system prioritizes and store the generated recommendations based on expected effectiveness, feasibility, and urgency, forming a ranked list.
\item \textit{Present Recommendations to Urban Area Manager} The Optimization Engine makes recommendations accessible through the Urban Area Manager Interface for further review and decision-making. The optimizations suggested must be readable for Urban Area Manager.
\end{itemize}
\textbf{Assumptions}: Traffic Management System can perform Machine Learning techniques and has enough computational power, data provided by the sensors are ready to be used, Traffic Management System provides optimiziation report for Urban Area Manager. \\
\textbf{Requirements}: the system shall generate optimization recommendations (FR1), the Optimization Engine shall present clearly readable and actionable optimization recommendations to the Urban Area Manager through a dedicated management dashboard interface (FR2).

\subsubsection*{Use Case 3: Review and Apply Changes}
\textbf{Primary Actors}: Urban Area Manager \\
\textbf{Supporting Actors}: Traffic Management System \\ 
\textbf{Use case flow}: 
\begin{itemize}
\item \textit{Access Optimization Recommendations} Urban Area Manager accesses the optimization suggestions presented by the Optimization Engine via the Management Dashboard.
\item \textit{Review Optimization Suggestions} Manager evaluates each optimization suggestion carefully, considering feasibility, expected benefits, potential impact, and citizen feedback previously received.
\item \textit{Approve or Reject Suggestions} Manager decides whether to approve or reject each suggestion individually.
\Item \textit{Implement Approved Changes} For each approved optimization, road direction changes are physically implemented and updated in traffic signs.
Public transit schedules are updated and communicated to the Public Transport Microservice.
\item \textit{Update City Maps and User App} All implemented changes are reflected promptly in: Official city maps.
The User App, ensuring citizens are immediately informed about modifications.
\item \textit{Publish Modification Reports} The Reporting Module generates and publishes a detailed public report via the User App, informing citizens about changes and their expected benefits.
\end{itemize}
\textbf{Assumptions}: the User App and city maps services are operational and capable of real-time updates.
\textbf{Requirements}: the system shall enable Urban Area Managers to approve or reject individual optimization suggestions explicitly, the system shall promptly update official city maps and the User App to reflect all implemented changes.

\subsection*{Type3 Use Cases}
\begin{table}[h!]
\centering
\begin{tabular}{|p{4.5cm}|p{8.5cm}|}
\hline
\textbf{Use Case} & \textbf{Description} \\
\hline
Event Classification & The system monitors official sources for events, categorizing them by size, location, and expected impact. \\
\hline
Traffic Disruption Forecasting & It predicts congestion using event data, historical trends, and real-time sensor input, enabling preemptive planning. \\
\hline
Dynamic Mobility Adaptation & The system adapts traffic lights, road usage, and public transport routes in response to forecasts. \\
\hline
Approval and Policy Coordination & Proposed actions are submitted to managers and city officials for validation and alignment with policy. \\
\hline
Public Engagement and Notification & Citizens are informed through mobile apps and dashboards about changes, alternate routes, and delays. \\
\hline
Post-Event Traffic Normalization & After events, the system restores standard configurations and analyzes effectiveness to improve future planning. \\
\hline
\end{tabular}
\caption{Type3 Use Cases}
\end{table}

\newpage

\subsection{Domain Assumptions}

The correct operation of the system is based on the following cross-cutting assumptions, organized by thematic area and referencing the relevant use case types (Type1, Type2, Type3).

\begin{itemize}

    \item \textit{Sensor Infrastructure Reliability} (Type1, Type2, Type3): The system assumes that traffic sensors deployed across the city, especially at intersections and major roads, are consistently operational. These sensors provide essential real-time data for adaptive traffic light control (Type1), daily traffic pattern analysis (Type2), and congestion prediction during large events (Type3).

    \item \textit{Autonomous Traffic Light Control} (Type1): The system is authorized to autonomously adjust traffic light timings based on real-time sensor data, without prior approval. These adjustments target immediate traffic imbalances and are the foundation for reactive traffic management.

    \item \textit{Urban Area Manager Oversight} (Type2, Type3): Although the system can autonomously manage traffic lights, larger changes, such as reconfiguring roads or updating public transport schedules, require approval from urban area managers. Their role ensures that data-driven suggestions align with long-term urban planning.

    \item \textit{Public Transport Data Availability} (Type2, Type3): The system relies on up-to-date access to the public transport microservice, providing accurate timetables and real-time data. This supports both optimization of daily operations (Type2) and event-based rerouting (Type3).

    \item \textit{Event Data Feeds} (Type3): The system assumes continuous access to official city event schedules, allowing early detection of high-impact gatherings and preparation of mobility strategies accordingly.

    \item \textit{Inter-System Communication} (Type 1, Type 2, Type 3): Reliable and stable network infrastructure is essential to enable seamless communication among the traffic management system, microservices, event monitoring module, sensors, and urban manager interfaces. Delays or failures in communication could hinder real-time reactions (Type1), the generation of optimization proposals (Type2), and the coordination of event-driven responses (Type3).

    \item \textit{Logging and Traceability} (Type1, Type2): All system actions, especially autonomous interventions and proposed suggestions, are assumed to be automatically logged. This ensures accountability, supports auditing by human actors, and enables future analysis.

    \item \textit{Real-Time Processing Capabilities} (Type1, Type3): The system is assumed to be capable of real-time or near-real-time decision-making, which is essential for reacting to live traffic conditions (Type1) and dynamically adapting to disruptions during events (Type3).

    \item \textit{Public Awareness and Engagement} (Type3): For event-driven adjustments to be effective, it is assumed that citizens actively consult and respond to the notifications distributed through the system. Informed commuters are critical to mitigating congestion during critical timeframes.
    
\end{itemize}

\newpage

\subsection{Requirements}

\subsubsection{Functional Requirements}

\subsubsection*{Type1}
These are mandatory actions the system must fulfill to meet its objectives:
\begin{itemize}
    \item \textbf{Real-Time Traffic Monitoring:}  
    The system must continuously collect data from traffic sensors installed at key intersections to monitor vehicle wait times and traffic density.

    \item \textbf{Traffic Imbalance Detection:}  
    The system must automatically detect significant traffic imbalances between intersecting roads using real-time sensor data.

    \item \textbf{Autonomous Light Timing Adjustment:}  
    The system must autonomously modify traffic light durations in response to detected imbalances, following pre-defined rules or adaptive algorithms.

    \item \textbf{Light Configuration Dispatch:}  
    The system must send updated traffic light timing configurations to the respective controllers without manual intervention.

    \item \textbf{Action Logging:}  
    Every automated change in traffic light behavior must be logged with timestamp, affected intersections, and rationale.

    \item \textbf{Daily Activity Summary:}  
    The system must generate a daily report detailing Type1 interventions and overall traffic conditions for transparency and analysis.
\end{itemize}

\subsubsection*{Type2}
These are mandatory actions the system must fulfill to meet its objectives:
\begin{itemize}
    \item \textbf{Traffic Data Analysis:}  
    The system must continuously collect and analyze traffic sensor data to detect daily traffic patterns and congestion areas.

    \item \textbf{Optimization Recommendation Generation:}  
    Based on traffic pattern analysis, the system must automatically propose targeted optimization actions. These include adjusting traffic light timing, redefining road directions (such as converting roads to one-way), and modifying public transportation schedules.

    \item \textbf{Suggestion Management:}  
    The system must provide urban area managers with an intuitive interface to view, review, and respond (approve/reject) to proposed optimizations.

    \item \textbf{Reporting:}  
    The system must regularly generate detailed reports documenting both proposed and implemented optimizations, including decisions made by urban area managers.
\end{itemize}

\subsubsection*{Type3}
The functional requirements are the mandatory actions that the system must carry out to meet its objective and illustrate the interactions between the system and the environment:
\begin{itemize}
    \item \textbf{Event-Driven Traffic Adjustments:}  
    The system must dynamically modify traffic configurations before and during events.

    \item \textbf{Predictive Congestion Forecasting:}  
    AI-driven forecasting enables early identification of mobility bottlenecks.

    \item \textbf{Approval Workflow Management:}  
    Proposed optimizations must undergo a structured review process.

    \item \textbf{Real-Time Public Transport Coordination:}  
    The system must adjust transport availability based on event-driven demand spikes.

    \item \textbf{Public Notification System:}  
    Real-time alerts must inform commuters of route modifications and alternative travel options.

    \item \textbf{Traffic Recovery and Analysis:}  
    The system must ensure post-event normalization and continuous refinement of optimization strategies.
\end{itemize}


\subsubsection{Non-Functional Requirements}

\subsubsection*{Type1}
These constraints or conditions influence how the system performs its functions:
\begin{itemize}
    \item \textbf{Real-Time Responsiveness:}  
    The system must analyze sensor input and apply traffic light adjustments with minimal latency to prevent congestion build-up.

    \item \textbf{Fault Tolerance:}  
    The system must handle temporary sensor or controller failures gracefully, continuing operation using fallback logic where needed.

    \item \textbf{Reliability and Availability:}  
    The traffic monitoring and control components must be continuously operational with high uptime to ensure uninterrupted flow management.

    \item \textbf{System Transparency:}  
    All autonomous decisions must be explainable and traceable via logs to support auditing and trust.

    \item \textbf{Data Integrity:}  
    Traffic sensor data and light timing commands must be transmitted securely and stored accurately to prevent corruption or manipulation.
\end{itemize}

\subsubsection*{Type2}
These constraints or conditions influence how the system performs its functions:
\begin{itemize}
    \item \textbf{Real-Time Responsiveness:}  
    The system must rapidly analyze traffic data and provide optimization suggestions without significant delays, enabling timely decisions by urban managers.

    \item \textbf{Scalability:}  
    It must efficiently handle the large volume of data collected from numerous traffic sensors distributed across an urban area, ensuring consistent performance as city infrastructure grows.

    \item \textbf{High Availability and Reliability:}  
    The system must be operational continuously (with minimal downtime) to support uninterrupted urban traffic management.

    \item \textbf{User-Friendly Interface:}  
    The interface used by urban area managers to review suggestions must be straightforward, intuitive, and accessible, facilitating quick and informed decision-making.

    \item \textbf{Security and Data Integrity:}  
    The system must secure sensitive traffic data and ensure that data used for analyses and recommendations remain accurate, protected from unauthorized access or manipulation.
\end{itemize}

\subsubsection*{Type3}
The non-functional requirements contain or characterize how the system performs its functions as a whole:
\begin{itemize}
    \item \textbf{Real-Time Responsiveness:}  
    The system must quickly analyze live traffic data and generate optimization suggestions with minimal delay, ensuring timely decision-making for urban managers and preventing congestion before it escalates.

    \item \textbf{Scalability:}  
    The system must handle fluctuations in urban mobility data, adapting to both routine commuting patterns and large-scale disruptions.

    \item \textbf{High Availability and Reliability:}  
    The system must operate continuously with minimal downtime to support uninterrupted urban traffic management.

    \item \textbf{User Accessibility:}  
    Notifications must be structured for easy comprehension by citizens, integrating multiple communication channels.

    \item \textbf{Security and Data Integrity:}  
    Unauthorized alterations must be prevented, ensuring optimization decisions are based on validated event forecasts.
\end{itemize}

\newpage

% SECTION 3 - System Design
\section{System Design}
This section outlines the architecture, sequence diagrams, key design decisions, and critical points relevant to event-driven urban mobility adaptations.

\subsection{Architectural Overview}
\subsection*{Type3}
The system follows a modular architecture, ensuring real-time adaptability to traffic disruptions caused by large-scale events.
\begin{itemize}
    \item \textbf{EventCollector} - Continuously monitors and retrieves official event data from external sources.
    \item \textbf{TrafficAnalyzer} - Processes congestion data, predicts affected areas, and identifies high-impact road segments.
    \item \textbf{ConfigurationGenerator} - Computes traffic adaptations, including signal adjustments, lane closures, and reroutes.
    \item \textbf{ApprovalModule} - Validates proposed modifications before execution by urban authorities.
    \item \textbf{NotificationService} - Publishes updates informing citizens about road closures, alternative routes, and adjusted transit schedules.
\end{itemize}

\newpage

\subsection{Sequence Diagrams}
\subsection*{Type3}
To ensure efficient operation, the system follows three main interaction sequences:

\begin{enumerate}
    \item {Event Reception}
    \begin{itemize}
        \item Event Collector detects an upcoming event.
        \item Event data is retrieved and logged in the system.
        \item TrafficAnalyzer prepares preliminary impact forecasts.
    \end{itemize}
    \item {Traffic Analysis and Configuration Generation}
    \begin{itemize}
        \item TrafficAnalyzer retrieves real-time congestion data.
        \item ConfigurationGenerator formulates optimized traffic control measures (e.g., signal adjustments, reroutes).
        \item The configurations are prepared for validation.
    \end{itemize}
    \item {Approval and Public Notification}
    \begin{itemize}
        \item The Urban Manager reviews proposed adjustments and approves the changes.
        \item NotificationService informs citizens of road closures, signal modifications, and public transport reroutes.
    \end{itemize}
\end{enumerate}

\newpage

\subsection{Critical points and design decisions}
\subsection*{Type3 - design decisions:}
\begin{itemize}
    \item Event-Driven Optimization --> Unlike Type 2 actions (which focus on routine traffic improvements), Type 3 actions react dynamically to event-based disruptions.
    \item Approval Workflow --> Ensures human oversight before implementing major transport reconfigurations.
    \item Public Engagement -->  Structured communication channels provide real-time notifications to affected commuters.
\end{itemize}
\subsection*{Type3 - critical aspects:}
\begin{itemize}
    \item Event Data Accuracy --> Must ensure data reliability from official event sources.
    \item Traffic Simulation Limitations --> Forecasting congestion trends must balance real-time responsiveness and computational efficiency.
    \item Citizen Adaptability --> The effectiveness of optimizations depends on commuter engagement with system-generated notifications.
\end{itemize}

\end{document}
