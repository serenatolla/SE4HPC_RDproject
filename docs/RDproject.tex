\documentclass[a4paper,12pt]{article}
\usepackage[utf8]{inputenc}
\usepackage{graphicx}
\usepackage{hyperref}
\usepackage{amsmath, amssymb}
\usepackage{authblk}


\title{Requirement Engineering and Design Project \\ SE4HPC_RDproject}
\vspace{3cm} % Adds space between title and authors

\author[1]{Giovanni La Gioia}
\vspace{0.5cm} % Space between authors
\author[2]{Luca Leonzio}
\vspace{0.5cm}
\author[3]{Matteo Parimbelli}
\vspace{0.5cm}
\author[4]{Serena Tolla}
\affil[1,2,3,4]{Politecnico di Milano}
\date{Last update: \today}

\begin{document}
\maketitle
\newpage
\tableofcontents
\clearpage
\newpage

% SECTION 1 - The Project and Its Goals
\section{The Project and Its Goals}
This document describes the \textbf{SustainCity} project, developed as part of the \textbf{Software Engineering for HPC} course. Its objective is to tackle urban mobility challenges using \textbf{real-time traffic management, event-driven optimizations, and public transport coordination}. By leveraging \textbf{dynamic infrastructure adaptations}, the system aims to \textbf{reduce congestion, improve sustainability}, and \textbf{enhance traffic flow efficiency} in urban areas.

We structured this section into three key parts:
\begin{itemize}
    \item The \textbf{Preface}, which introduces the broader environmental and urban challenges motivating this project.
    \item The \textbf{Problem Posed}, detailing the technical and societal issues to be addressed.
    \item The \textbf{Project Goals}, outlining the core functionalities and expected outcomes.
\end{itemize}

\subsection{Preface}
Urban mobility inefficiencies significantly contribute to \textbf{air pollution, congestion, and greenhouse gas emissions}. Despite the presence of \textbf{public transportation}, many cities still face considerable challenges in \textbf{traffic optimization} and \textbf{event-based urban planning}.

To address these concerns, the \textbf{SustainCity} project integrates \textbf{traffic monitoring, event planning, and transport analytics} into an adaptive system capable of making \textbf{data-driven recommendations} for \textbf{real-time city infrastructure improvements}.

\subsection{Problem Posed}
The \textbf{SustainCity} system aims to optimize urban traffic using three key strategies:

\begin{itemize}
    \item \textbf{Type1 Actions (Real-time Traffic Optimization)} 
    The system dynamically adjusts traffic light durations based on live congestion data.  

    \item \textbf{Type2 Actions (Long-term Traffic Pattern Analysis)}  
    The system analyzes daily congestion patterns** to recommend permanent adjustments, such as one-way street configurations, adaptive traffic signal schedules, and public transport optimizations.

    \item \textbf{Type3 Actions (Event-driven Traffic Management)}
    SustainCity monitors city event feeds (e.g., concerts, fairs, sports games) and adjusts road and transport infrastructure dynamically to reduce congestion during high-footfall gatherings.
\end{itemize}

\noindent Additionally, SustainCity generates \textbf{reports for urban planners and citizens}:
\begin{itemize}
    \item \textbf{Daily Reports}: Summarizing average traffic flows and actions taken (Type1).
    \item \textbf{Yearly Reports}: Documenting major infrastructure proposals (Type2, Type3) and their implementation status (accepted/rejected).
\end{itemize}

\subsection{Project Goals}
To improve urban traffic management, SustainCity focuses on:
\begin{itemize}
    \item \textbf{Optimizing real-time traffic signals} using congestion data (Type1).
    \item \textbf{Enhancing urban infrastructure and transport schedules} (Type2).
    \item \textbf{Minimizing congestion during major city events} (Type3).
    \item \textbf{Generating actionable insights for city planners and the public}.
\end{itemize}

The project is designed to be \textbf{scalable, adaptive, and fully integrated} with existing urban mobility systems. By providing \textbf{data-driven recommendations}, SustainCity aims to create \textbf{efficient, sustainable urban transport solutions}.

\newpage

% SECTION 2 - Requirement Analysis
\section{Requirement Analysis}

\subsection{Relevant Human and Non-Human Actors}

\subsection{Use Cases}


\subsection{Domain Assumptions}


\subsection{Requirements}
\subsubsection{Functional Requirements}

\subsubsection{Non-Functional Requirements}

\newpage 
\subtitle{Type 2}

\section{Requirement Analysis}

\subsection{Relevant Human and Non-Human Actors}

\begin{itemize}
    \item \textbf{Urban Area Manager}:
Responsible for managing urban infrastructure and ensuring efficient traffic flow, this actor evaluates suggestions provided by the system. The manager holds authority to approve or reject recommendations for changing traffic configurations, public transport scheduling, and road directions. Their decision-making directly influences the effectiveness and practicality of proposed optimizations.

\item \textbf{Traffic Management System}:
Acts as the core analytical component by collecting, processing, and interpreting data from various sources, such as traffic sensors. This actor continuously monitors daily traffic flows, identifies congestion hotspots, and detects recurring traffic patterns that could benefit from optimization.

\item \textbf{Public Transport Microservice}:
Provides comprehensive and real-time public transport schedules. By integrating this microservice, the system can propose modifications to transport schedules, aligning them with identified traffic patterns to optimize commuter convenience and reduce congestion.

\item \textbf{Citizens}:
Although not directly interacting with the internal workings of the system, citizens are vital end-users who indirectly benefit from the optimizations. They receive publicly accessible reports about implemented and proposed changes, leading to improved commuting experiences through reduced congestion and more predictable traffic flows.

\item \textbf{Sensors}:
Physical devices deployed at intersections and major roadways, these actors gather real-time data about traffic conditions. Accurate and timely sensor data is foundational, enabling the system to reliably analyze traffic patterns and propose meaningful optimizations.
\end{itemize}


\subsection{Use Cases}

\begin{itemize}
    \item \textbf{Traffic Pattern Analysis}:
The system continuously processes daily traffic data gathered from city-wide sensors to detect patterns and identify areas frequently experiencing congestion or inefficient traffic flow. By understanding traffic behaviors over time, it can pinpoint critical issues, such as bottlenecks caused by suboptimal traffic light timing, inefficient road layouts, or misaligned public transport schedules. Recognizing these patterns is essential for accurately suggesting targeted interventions.

\item \textbf{Suggest Optimizations}:
Utilizing insights gained from the traffic pattern analysis, the system generates actionable recommendations aimed at reducing congestion and improving overall commuting efficiency. These recommendations include adjusting the duration and synchronization of traffic lights, revising street directions (e.g., converting roads into one-way streets), and modifying public transportation schedules to better align with commuter patterns. These suggestions are then systematically communicated to urban area managers for review.

\item \textbf{Review and Apply Changes}:
In this use case, urban area managers critically examine the optimization suggestions provided by the system. They evaluate the practicality, potential impact, and feasibility of each recommendation before making decisions. Approved suggestions are subsequently implemented within the city’s infrastructure, directly influencing urban commuting conditions. Suggestions deemed impractical or undesirable are rejected, with the decision outcomes logged and documented for reporting purposes.
\end{itemize}

\subsection{Domain Assumptions}

The following assumptions establish foundational conditions upon which the system's functionality depends:

\begin{itmeize}

\item \textbf{Reliable Sensor Infrastructure}:
It is assumed that sensors for measuring traffic flow are consistently operational, providing accurate and timely data. Reliable sensor data is fundamental to correctly identifying daily traffic patterns and making valid optimization suggestions.

\item \textbf{Availability of Public Transport Data}:
The system presumes continuous access to an up-to-date public transport microservice. Accurate schedules and real-time information from this microservice are critical for suggesting effective adjustments to public transportation based on identified traffic patterns.

\item \textbf{Authority of Urban Area Managers}:
It is assumed that urban area managers hold exclusive authority for approving or rejecting optimization suggestions. The system does not autonomously apply changes (other than traffic lights for Type 1 actions), ensuring that any significant modifications are subject to human oversight and final decision-making.

\item \textbf{Stable Communication Infrastructure}:
The system assumes stable and continuous network connectivity between all components (traffic management system, sensors, microservices, and urban area manager interfaces), facilitating efficient data transfer and timely decision-making.

\end{itmeize}

\subsection{Requirements}
\subsubsection{Functional Requirements}

These are mandatory actions the system must fulfill to meet its objectives:

\begin{itemize}
    
\item Traffic Data Analysis:
The system must continuously collect and analyze traffic sensor data to detect daily traffic patterns and congestion areas.

\item Optimization Recommendation Generation:
Based on traffic pattern analysis, the system must automatically propose targeted optimization actions. These include adjusting traffic light timing, redefining road directions (such as converting roads to one-way), and modifying public transportation schedules.

\item Suggestion Management:
The system must provide urban area managers with an intuitive interface to view, review, and respond (approve/reject) to proposed optimizations.

\item Reporting:
The system must regularly generate detailed reports documenting both proposed and implemented optimizations, including decisions made by urban area managers.

\end{itemize}

\subsubsection{Non-Functional Requirements}
\begin{itemize}
These constraints or conditions influence how the system performs its functions:

\item Real-Time Responsiveness:
The system must rapidly analyze traffic data and provide optimization suggestions without significant delays, enabling timely decisions by urban managers.

\item Scalability:
It must efficiently handle the large volume of data collected from numerous traffic sensors distributed across an urban area, ensuring consistent performance as city infrastructure grows.

\item High Availability and Reliability:
The system must be operational continuously (with minimal downtime) to support uninterrupted urban traffic management.

\item User-Friendly Interface:
The interface used by urban area managers to review suggestions must be straightforward, intuitive, and accessible, facilitating quick and informed decision-making.

\item Security and Data Integrity:
The system must secure sensitive traffic data and ensure that data used for analyses and recommendations remain accurate, protected from unauthorized access or manipulation.
    
\end{itemize}

\newpage

% SECTION 3 - Design
\section{Design}
\subsection{General Description of the Architecture}

\subsection{Sequence Diagrams}



\subsection{Critical Points and Design Decisions}


\end{document}
