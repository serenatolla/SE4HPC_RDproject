\documentclass[a4paper,12pt]{article}
\usepackage[utf8]{inputenc}
\usepackage{graphicx}
\usepackage{hyperref}
\usepackage{amsmath, amssymb}
\usepackage{authblk}


\title{Requirement Engineering and Design Project \\ SE4HPC_RDproject} 
\author[1]{Giovanni La Gioia}
\author[2]{Luca Leonzio}
\author[3]{Matteo Parimbelli}
\author[4]{Serena Tolla}
\affil[1,2,3]{Politecnico di Milano}
\date{\today}

\begin{document}

\maketitle
\tableofcontents
\newpage

% SEZIONE 1 - Introduzione
\section{Introduzione}
Questo documento descrive il progetto \textbf{SustainCity}, sviluppato per il corso di \textit{Software Engineering for HPC}. 
L'obiettivo è analizzare la gestione della mobilità urbana per ottimizzarla in termini di sostenibilità, riducendo traffico e impatti ambientali.

\subsection{Obiettivi}
Gli obiettivi principali del progetto sono:
\begin{itemize}
    \item Monitorare i flussi di traffico e adattare la durata dei semafori dinamicamente.
    \item Analizzare schemi di traffico per ottimizzare la viabilità e i trasporti pubblici.
    \item Gestire eventi che attirano grandi folle per minimizzare congestioni stradali.
\end{itemize}

\newpage

% SEZIONE 2 - Analisi dei requisiti
\section{Analisi dei requisiti}
\subsection{Attori coinvolti}
Gli attori principali nel sistema includono:
\begin{itemize}
    \item \textbf{Sensori del traffico}, che misurano i tempi di attraversamento agli incroci.
    \item \textbf{Microservizi di trasporto pubblico}, che forniscono orari e disponibilità.
    \item \textbf{Canali informativi}, che trasmettono notizie sugli eventi urbani.
    \item \textbf{Cittadini e urban area managers}, coinvolti nell'analisi e applicazione delle soluzioni.
\end{itemize}

\subsection{Requisiti funzionali}
\begin{itemize}
    \item Il sistema deve ricevere dati aggiornati sui flussi di traffico.
    \item Deve analizzare i pattern e proporre ottimizzazioni sulle configurazioni stradali.
    \item Deve notificare i gestori urbani sui possibili interventi.
\end{itemize}

\newpage

% SEZIONE 3 - Architettura del sistema
\section{Architettura del sistema}
\subsection{Diagrammi dei componenti}
Il sistema è composto da diverse unità, tra cui:
\begin{itemize}
    \item \textbf{Modulo di analisi del traffico}, che raccoglie dati e identifica schemi.
    \item \textbf{Modulo di configurazione semaforica}, che adatta la durata dei semafori.
    \item \textbf{Modulo di gestione eventi}, che suggerisce modifiche viarie basate sugli eventi.
\end{itemize}

\subsection{Diagrammi di sequenza}
Per ogni azione chiave del sistema, vengono definiti i diagrammi di sequenza che descrivono l'interazione tra le componenti.

\newpage

% SEZIONE 4 - Implementazione e risultati attesi
\section{Implementazione e risultati attesi}
L'implementazione sarà gestita attraverso una repository GitHub organizzata con le seguenti cartelle:
\begin{itemize}
    \item \textbf{docs/} - Contiene il documento LaTeX e le specifiche del progetto.
    \item \textbf{src/} - Include il codice sorgente scritto in C++.
    \item \textbf{diagrams/} - Contiene i diagrammi UML e altri schemi tecnici.
\end{itemize}

\subsection{Metriche di valutazione}
Il successo del sistema sarà valutato in base ai seguenti parametri:
\begin{itemize}
    \item Riduzione della congestione stradale.
    \item Ottimizzazione delle risorse di trasporto pubblico.
    \item Accettazione delle proposte di configurazione da parte delle autorità urbane.
\end{itemize}

\end{document}

